\documentclass[12pt,]{article}
\usepackage{lmodern}
\usepackage{amssymb,amsmath}
\usepackage{ifxetex,ifluatex}
\usepackage{fixltx2e} % provides \textsubscript
\ifnum 0\ifxetex 1\fi\ifluatex 1\fi=0 % if pdftex
  \usepackage[T1]{fontenc}
  \usepackage[utf8]{inputenc}
\else % if luatex or xelatex
  \ifxetex
    \usepackage{mathspec}
  \else
    \usepackage{fontspec}
  \fi
  \defaultfontfeatures{Ligatures=TeX,Scale=MatchLowercase}
\fi
% use upquote if available, for straight quotes in verbatim environments
\IfFileExists{upquote.sty}{\usepackage{upquote}}{}
% use microtype if available
\IfFileExists{microtype.sty}{%
\usepackage{microtype}
\UseMicrotypeSet[protrusion]{basicmath} % disable protrusion for tt fonts
}{}
\usepackage[margin=1in]{geometry}
\usepackage[unicode=true]{hyperref}
\hypersetup{
            pdftitle={AMS 274 - GLM HW 2},
            pdfauthor={Arthur Lui},
            pdfborder={0 0 0},
            breaklinks=true}
\urlstyle{same}  % don't use monospace font for urls
\usepackage{color}
\usepackage{fancyvrb}
\newcommand{\VerbBar}{|}
\newcommand{\VERB}{\Verb[commandchars=\\\{\}]}
\DefineVerbatimEnvironment{Highlighting}{Verbatim}{commandchars=\\\{\}}
% Add ',fontsize=\small' for more characters per line
\newenvironment{Shaded}{}{}
\newcommand{\KeywordTok}[1]{\textcolor[rgb]{0.00,0.44,0.13}{\textbf{{#1}}}}
\newcommand{\DataTypeTok}[1]{\textcolor[rgb]{0.56,0.13,0.00}{{#1}}}
\newcommand{\DecValTok}[1]{\textcolor[rgb]{0.25,0.63,0.44}{{#1}}}
\newcommand{\BaseNTok}[1]{\textcolor[rgb]{0.25,0.63,0.44}{{#1}}}
\newcommand{\FloatTok}[1]{\textcolor[rgb]{0.25,0.63,0.44}{{#1}}}
\newcommand{\ConstantTok}[1]{\textcolor[rgb]{0.53,0.00,0.00}{{#1}}}
\newcommand{\CharTok}[1]{\textcolor[rgb]{0.25,0.44,0.63}{{#1}}}
\newcommand{\SpecialCharTok}[1]{\textcolor[rgb]{0.25,0.44,0.63}{{#1}}}
\newcommand{\StringTok}[1]{\textcolor[rgb]{0.25,0.44,0.63}{{#1}}}
\newcommand{\VerbatimStringTok}[1]{\textcolor[rgb]{0.25,0.44,0.63}{{#1}}}
\newcommand{\SpecialStringTok}[1]{\textcolor[rgb]{0.73,0.40,0.53}{{#1}}}
\newcommand{\ImportTok}[1]{{#1}}
\newcommand{\CommentTok}[1]{\textcolor[rgb]{0.38,0.63,0.69}{\textit{{#1}}}}
\newcommand{\DocumentationTok}[1]{\textcolor[rgb]{0.73,0.13,0.13}{\textit{{#1}}}}
\newcommand{\AnnotationTok}[1]{\textcolor[rgb]{0.38,0.63,0.69}{\textbf{\textit{{#1}}}}}
\newcommand{\CommentVarTok}[1]{\textcolor[rgb]{0.38,0.63,0.69}{\textbf{\textit{{#1}}}}}
\newcommand{\OtherTok}[1]{\textcolor[rgb]{0.00,0.44,0.13}{{#1}}}
\newcommand{\FunctionTok}[1]{\textcolor[rgb]{0.02,0.16,0.49}{{#1}}}
\newcommand{\VariableTok}[1]{\textcolor[rgb]{0.10,0.09,0.49}{{#1}}}
\newcommand{\ControlFlowTok}[1]{\textcolor[rgb]{0.00,0.44,0.13}{\textbf{{#1}}}}
\newcommand{\OperatorTok}[1]{\textcolor[rgb]{0.40,0.40,0.40}{{#1}}}
\newcommand{\BuiltInTok}[1]{{#1}}
\newcommand{\ExtensionTok}[1]{{#1}}
\newcommand{\PreprocessorTok}[1]{\textcolor[rgb]{0.74,0.48,0.00}{{#1}}}
\newcommand{\AttributeTok}[1]{\textcolor[rgb]{0.49,0.56,0.16}{{#1}}}
\newcommand{\RegionMarkerTok}[1]{{#1}}
\newcommand{\InformationTok}[1]{\textcolor[rgb]{0.38,0.63,0.69}{\textbf{\textit{{#1}}}}}
\newcommand{\WarningTok}[1]{\textcolor[rgb]{0.38,0.63,0.69}{\textbf{\textit{{#1}}}}}
\newcommand{\AlertTok}[1]{\textcolor[rgb]{1.00,0.00,0.00}{\textbf{{#1}}}}
\newcommand{\ErrorTok}[1]{\textcolor[rgb]{1.00,0.00,0.00}{\textbf{{#1}}}}
\newcommand{\NormalTok}[1]{{#1}}
\usepackage{longtable,booktabs}
\IfFileExists{parskip.sty}{%
\usepackage{parskip}
}{% else
\setlength{\parindent}{0pt}
\setlength{\parskip}{6pt plus 2pt minus 1pt}
}
\setlength{\emergencystretch}{3em}  % prevent overfull lines
\providecommand{\tightlist}{%
  \setlength{\itemsep}{0pt}\setlength{\parskip}{0pt}}
\setcounter{secnumdepth}{0}
% Redefines (sub)paragraphs to behave more like sections
\ifx\paragraph\undefined\else
\let\oldparagraph\paragraph
\renewcommand{\paragraph}[1]{\oldparagraph{#1}\mbox{}}
\fi
\ifx\subparagraph\undefined\else
\let\oldsubparagraph\subparagraph
\renewcommand{\subparagraph}[1]{\oldsubparagraph{#1}\mbox{}}
\fi
\usepackage{bm}
\usepackage{bbm}
\usepackage{xifthen}
\pagestyle{empty}
\newcommand{\norm}[1]{\left\lVert#1\right\rVert}
\newcommand{\p}[1]{\left(#1\right)}
\newcommand{\bk}[1]{\left[#1\right]}
\newcommand{\bc}[1]{ \left\{#1\right\} }
\newcommand{\abs}[1]{ \left|#1\right| }
\newcommand{\mat}{ \begin{pmatrix} }
\newcommand{\tam}{ \end{pmatrix} }
\newcommand{\suml}{ \sum_{i=1}^n }
\newcommand{\prodl}{ \prod_{i=1}^n }
\newcommand{\ds}{ \displaystyle }
\newcommand{\df}[2]{ \frac{d#1}{d#2} }
\newcommand{\ddf}[2]{ \frac{d^2#1}{d{#2}^2} }
\newcommand{\pd}[2]{ \frac{\partial#1}{\partial#2} }
\newcommand{\pdd}[2]{ \frac{\partial^2#1}{\partial{#2}^2} }
\newcommand{\N}{ \mathcal{N} }
\newcommand{\E}{ \text{E} }
\newcommand{\V}{ \text{Var} }
\newcommand{\Poisson}{\text{Poisson}}

\title{AMS 274 - GLM HW 2}
\author{Arthur Lui}
\date{27 October, 2016}

\begin{document}
\maketitle

\allowdisplaybreaks

\subsection{1a)}\label{a}

For \(y_i \sim \Poisson(\mu_i)\), the probability density is
\(f(y_i|\mu_i) = \ds\frac{e^{-\mu_i}\mu_i^{y_i}}{y_i!}\). Also, the
Exponential Dispersion Family (EDF) has pdf
\(p(y_i|\theta_i,\phi) = \exp\bc{\ds\frac{y_i\theta_i - b(\theta_i)}{\phi/w_i} + c(y_i,\phi)}\).
And the deviance statistic is defined as \[
D = 2\suml w_i \bc{y_i\p{\tilde{\theta_i}-\hat{\theta_i}} - b(\tilde{\theta_i}) + b(\hat{\theta_i})}.\\
\]

First, note that the Poisson distribution is a member of the EDF with \[
\begin{split}
\theta_i &= \log(\mu_i)\\
b(\theta_i) &= \exp(\theta_i) = -\mu_i\\
w_i = \phi &= 1
\end{split}
\] So,

\begin{align*}
D &= 2\suml \bc{y_i\p{\tilde{\theta_i}-\hat{\theta_i}} - b(\tilde{\theta_i}) + b(\hat{\theta_i})} \\
  &= 2\suml \bc{y_i\bc{\log(\tilde{\mu_i})-\log(\hat{\mu_i})} + \tilde{\mu_i} - \hat{\mu_i}} \\
  &= 2\suml \bc{y_i\log\p{\frac{\tilde{\mu_i}}{\hat{\mu_i}}} + \tilde{\mu_i} - \hat{\mu_i}} \\
  &= 2\suml \bc{y_i\log\p{\frac{y_i}{\hat{\mu_i}}} + y_i - \hat{\mu_i}} \\
  &= 2\suml \bc{y_i\log\p{\frac{y_i}{\hat{\mu_i}}}} + 2\suml\bc{y_i - \hat{\mu_i}}\\
\end{align*}

where \(\hat{\mu_i} = g^{-1}(x_i^T\hat\beta)\), and \(\hat\beta\) is the
MLE of \(\beta\) under the reduced model.

\subsection{1b)}\label{b}

In the special case where \(g(\cdot) = \log(\cdot)\), we have
\(\hat{\mu_i} = g^{-1}(x_i^T\hat\beta) = \exp(x_i^T\hat\beta)\), and the
deviance is

\begin{align*}
D &= 2\suml \bc{y_i\log\p{\frac{y_i}{\hat{\mu_i}}}} + 2\suml\bc{y_i - \hat{\mu_i}}\\
  &= 2\suml \bc{y_i\log\p{\frac{y_i}{\hat{\mu_i}}}} + 2\suml\bc{y_i - \exp(x_i^T\hat\beta)}\\
  &= 2\suml \bc{y_i\log\p{\frac{y_i}{\hat{\mu_i}}}} + 2n\bar{y} - 2\suml\exp(x_i^T\hat\beta)\\
  &= 2\suml y_i\log\p{\frac{y_i}{\hat{\mu_i}}} \\
\end{align*}

\subsection{2a)}\label{a-1}

\begin{align*}
f(y_i|\mu_i,\nu) &= \ds\frac{(\nu/\mu_i)^\nu y_i^{\nu-1}}{\Gamma(\nu)}\exp(-\nu y_i/\mu_i)\\
&= \exp\bc{-\frac{\nu y_i}{\mu_i} + (\nu-1)\log y_i + \nu\log\frac{\nu}{\mu_i}-\log\Gamma(\nu)}\\
&= \exp\bc{\ds\frac{y_i \mu_i^{-1} - \log\ds\frac{\mu_i}{\nu}}{-\nu^{-1}} + (\nu-1)\log(y_i) - \log\Gamma(\nu)}
\end{align*}

which is a member of the EDF with

\begin{align*}
\theta_i &= \mu_i^{-1} \\
b(\theta_i) &= \log\frac{\mu_i}{\nu} = -\log(\nu\theta_i)\\
w_i &= -1 \\
\phi &= \nu^{-1} \\
\end{align*}

\newpage

\subsection{2b)}\label{b-1}

The scaled deviance is

\begin{align*}
D^* &= \frac{2}{\phi}\suml w_i\bc{y_i\p{\tilde{\theta_i}-\hat{\theta_i}} - b(\tilde{\theta_i}) + b(\hat{\theta_i})} \\
    &= -\frac{2}{\phi}\suml \bc{y_i\p{\tilde{\mu_i}^{-1}-\hat{\mu_i}^{-1}} - \log\frac{\tilde{\mu_i}}{\nu} + \log\frac{\hat{\mu_i}}{\nu}} \\
    &= -\frac{2}{\phi}\suml \bc{y_i\p{\tilde{\mu_i}^{-1}-\hat{\mu_i}^{-1}} + \log\frac{\hat{\mu_i}}{\tilde{\mu_i}} }\\
    &= -\frac{2}{\phi}\suml \bc{y_i\p{{y_i}^{-1}-\hat{\mu_i}^{-1}} + \log(y_i\hat\mu_i) }\\
    &= -\frac{2}{\phi}\suml 1-\frac{y_i}{\hat{\mu_i}} + \log(y_i\hat\mu_i)\\
    &= \frac{1}{\phi} D
\end{align*}

where
\(D = 2\ds\suml\bc{\frac{y_i}{\hat{\mu_i}} - \log(y_i\hat\mu_i) -1}\) is
the deviance, and \(\hat{\mu_i} = g^{-1}(x_i^T\hat\beta)\).

\subsection{3a)}\label{a-2}

\begin{verbatim}
Call:
glm(formula = faults ~ length, family = "poisson", data = fabric)

Deviance Residuals:
     Min        1Q    Median        3Q       Max
-2.74127  -1.13312  -0.03904   0.66179   3.07446

Coefficients:
             Estimate Std. Error z value Pr(>|z|)
(Intercept) 0.9717506  0.2124693   4.574 4.79e-06 ***
length      0.0019297  0.0003063   6.300 2.97e-10 ***
---
Signif. codes:  0 ‘***’ 0.001 ‘**’ 0.01 ‘*’ 0.05 ‘.’ 0.1 ‘ ’ 1

(Dispersion parameter for poisson family taken to be 1)

    Null deviance: 103.714  on 31  degrees of freedom
Residual deviance:  61.758  on 30  degrees of freedom
AIC: 189.06

Number of Fisher Scoring iterations: 4
\end{verbatim}

\subsection{3b)}\label{b-2}

\begin{verbatim}
Call:
glm(formula = faults ~ length, family = "quasipoisson", data = fabric)

Deviance Residuals:
     Min        1Q    Median        3Q       Max
-2.74127  -1.13312  -0.03904   0.66179   3.07446

Coefficients:
             Estimate Std. Error t value Pr(>|t|)
(Intercept) 0.9717506  0.3095033   3.140 0.003781 **
length      0.0019297  0.0004462   4.325 0.000155 ***
---
Signif. codes:  0 ‘***’ 0.001 ‘**’ 0.01 ‘*’ 0.05 ‘.’ 0.1 ‘ ’ 1

(Dispersion parameter for quasipoisson family taken to be 2.121965)

    Null deviance: 103.714  on 31  degrees of freedom
Residual deviance:  61.758  on 30  degrees of freedom
AIC: NA

Number of Fisher Scoring iterations: 4
\end{verbatim}

\subsection{3c)}\label{c}

\paragraph{Likelihood:}\label{likelihood}

\begin{align*}
\hat{\eta_0} &= x_0^T\hat{\beta} \\
\\
\E\bk{\hat{\eta_0}} &= \E\bk{x_0^T\hat\beta}\\
&= x_0^T\E\bk{\hat\beta}\\
&= x_0^T\beta \\
\\
\V(\hat{\eta_0}) &= \V(x_0^T\hat\beta) \\
&= \V(x_0^T\hat\beta) \\
&= x_0^T\V(\hat\beta)x_0 \\
&= x_0^TJ^{-1}(\beta)x_0 \\
\\
\end{align*}

Therefore, a point estimate for \(\eta_0\) is \(\hat{\eta_0}\), and an
interval estimate is
\(x_0^T\hat\beta \pm z_{.025} \sqrt{x_0^T J^{-1}(\hat\beta)x_0}\).

\subsection{3b)}\label{b-3}

\begin{align*}
\tilde{\eta_0} &= x_0^T\tilde{\beta} \\
\\
\E\bk{\tilde{\eta_0}} &= x_0^T\beta \\
\\
\V(\hat{\eta_0}) &= \V(x_0^T\hat\beta) \\
&= x_0^T\V(\hat\beta)x_0 \\
&= x_0^TJ^{-1}(\beta,\tilde{\phi})x_0 \\
&= x_0^T\p{J(\beta)/\tilde{\phi}}^{-1}x_0 \\
&= \tilde{\phi}x_0^T J^{-1}(\beta)x_0 \\
\\
\end{align*}

Therefore, a point estimate for \(\eta_0\) is \(\tilde{\eta_0}\), and an
interval estimate is
\(x_0^T\tilde\beta \pm z_{.025} \sqrt{\tilde\phi x_0^T J^{-1}(\hat\beta)x_0}\).

\begin{longtable}[]{@{}clcc@{}}
\toprule
\begin{minipage}[b]{0.11\columnwidth}\centering\strut
\(x_0\)\strut
\end{minipage} & \begin{minipage}[b]{0.18\columnwidth}\raggedright\strut
Model\strut
\end{minipage} & \begin{minipage}[b]{0.21\columnwidth}\centering\strut
Point Estimate\strut
\end{minipage} & \begin{minipage}[b]{0.27\columnwidth}\centering\strut
Interval Estimate\strut
\end{minipage}\tabularnewline
\midrule
\endhead
\begin{minipage}[t]{0.11\columnwidth}\centering\strut
500\strut
\end{minipage} & \begin{minipage}[t]{0.18\columnwidth}\raggedright\strut
Poisson\strut
\end{minipage} & \begin{minipage}[t]{0.21\columnwidth}\centering\strut
1.936624\strut
\end{minipage} & \begin{minipage}[t]{0.27\columnwidth}\centering\strut
(1.783437, 2.089811)\strut
\end{minipage}\tabularnewline
\begin{minipage}[t]{0.11\columnwidth}\centering\strut
500\strut
\end{minipage} & \begin{minipage}[t]{0.18\columnwidth}\raggedright\strut
Quasi-poisson\strut
\end{minipage} & \begin{minipage}[t]{0.21\columnwidth}\centering\strut
1.936624\strut
\end{minipage} & \begin{minipage}[t]{0.27\columnwidth}\centering\strut
(1.713477, 2.159771)\strut
\end{minipage}\tabularnewline
\begin{minipage}[t]{0.11\columnwidth}\centering\strut
995\strut
\end{minipage} & \begin{minipage}[t]{0.18\columnwidth}\raggedright\strut
Poisson\strut
\end{minipage} & \begin{minipage}[t]{0.21\columnwidth}\centering\strut
2.891849\strut
\end{minipage} & \begin{minipage}[t]{0.27\columnwidth}\centering\strut
(2.662679, 3.121018)\strut
\end{minipage}\tabularnewline
\begin{minipage}[t]{0.11\columnwidth}\centering\strut
995\strut
\end{minipage} & \begin{minipage}[t]{0.18\columnwidth}\raggedright\strut
Quasi-poisson\strut
\end{minipage} & \begin{minipage}[t]{0.21\columnwidth}\centering\strut
2.891849\strut
\end{minipage} & \begin{minipage}[t]{0.27\columnwidth}\centering\strut
(2.558018, 3.225679)\strut
\end{minipage}\tabularnewline
\bottomrule
\end{longtable}

\begin{Shaded}
\begin{Highlighting}[]
\KeywordTok{val} \NormalTok{x = }\DecValTok{1}
\KeywordTok{val} \NormalTok{l = List(}\DecValTok{1}\NormalTok{,}\DecValTok{2}\NormalTok{,}\DecValTok{3}\NormalTok{)}
\KeywordTok{val} \NormalTok{s = }\StringTok{"asd"}
\KeywordTok{if} \NormalTok{(x < }\DecValTok{1}\NormalTok{) \{}
  \NormalTok{x }
\NormalTok{\} }\KeywordTok{else} \NormalTok{b}
\end{Highlighting}
\end{Shaded}

\end{document}
